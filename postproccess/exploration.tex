\documentclass[]{article}
\usepackage{lmodern}
\usepackage{amssymb,amsmath}
\usepackage{ifxetex,ifluatex}
\usepackage{fixltx2e} % provides \textsubscript
\ifnum 0\ifxetex 1\fi\ifluatex 1\fi=0 % if pdftex
  \usepackage[T1]{fontenc}
  \usepackage[utf8]{inputenc}
\else % if luatex or xelatex
  \ifxetex
    \usepackage{mathspec}
  \else
    \usepackage{fontspec}
  \fi
  \defaultfontfeatures{Ligatures=TeX,Scale=MatchLowercase}
\fi
% use upquote if available, for straight quotes in verbatim environments
\IfFileExists{upquote.sty}{\usepackage{upquote}}{}
% use microtype if available
\IfFileExists{microtype.sty}{%
\usepackage{microtype}
\UseMicrotypeSet[protrusion]{basicmath} % disable protrusion for tt fonts
}{}
\usepackage[margin=1in]{geometry}
\usepackage{hyperref}
\hypersetup{unicode=true,
            pdftitle={JF\_processing},
            pdfborder={0 0 0},
            breaklinks=true}
\urlstyle{same}  % don't use monospace font for urls
\usepackage{graphicx,grffile}
\makeatletter
\def\maxwidth{\ifdim\Gin@nat@width>\linewidth\linewidth\else\Gin@nat@width\fi}
\def\maxheight{\ifdim\Gin@nat@height>\textheight\textheight\else\Gin@nat@height\fi}
\makeatother
% Scale images if necessary, so that they will not overflow the page
% margins by default, and it is still possible to overwrite the defaults
% using explicit options in \includegraphics[width, height, ...]{}
\setkeys{Gin}{width=\maxwidth,height=\maxheight,keepaspectratio}
\IfFileExists{parskip.sty}{%
\usepackage{parskip}
}{% else
\setlength{\parindent}{0pt}
\setlength{\parskip}{6pt plus 2pt minus 1pt}
}
\setlength{\emergencystretch}{3em}  % prevent overfull lines
\providecommand{\tightlist}{%
  \setlength{\itemsep}{0pt}\setlength{\parskip}{0pt}}
\setcounter{secnumdepth}{0}
% Redefines (sub)paragraphs to behave more like sections
\ifx\paragraph\undefined\else
\let\oldparagraph\paragraph
\renewcommand{\paragraph}[1]{\oldparagraph{#1}\mbox{}}
\fi
\ifx\subparagraph\undefined\else
\let\oldsubparagraph\subparagraph
\renewcommand{\subparagraph}[1]{\oldsubparagraph{#1}\mbox{}}
\fi

%%% Use protect on footnotes to avoid problems with footnotes in titles
\let\rmarkdownfootnote\footnote%
\def\footnote{\protect\rmarkdownfootnote}

%%% Change title format to be more compact
\usepackage{titling}

% Create subtitle command for use in maketitle
\newcommand{\subtitle}[1]{
  \posttitle{
    \begin{center}\large#1\end{center}
    }
}

\setlength{\droptitle}{-2em}

  \title{JF\_processing}
    \pretitle{\vspace{\droptitle}\centering\huge}
  \posttitle{\par}
    \author{}
    \preauthor{}\postauthor{}
    \date{}
    \predate{}\postdate{}
  
\usepackage{booktabs}
\usepackage{longtable}
\usepackage{array}
\usepackage{multirow}
\usepackage{wrapfig}
\usepackage{float}
\usepackage{colortbl}
\usepackage{pdflscape}
\usepackage{tabu}
\usepackage{threeparttable}
\usepackage{threeparttablex}
\usepackage[normalem]{ulem}
\usepackage{makecell}
\usepackage{xcolor}

\begin{document}
\maketitle

\includegraphics{exploration_files/figure-latex/unnamed-chunk-1-1.pdf}

\begin{center}\rule{0.5\linewidth}{\linethickness}\end{center}

\begin{center}\rule{0.5\linewidth}{\linethickness}\end{center}

\subsection{Carbon}\label{carbon}

\paragraph{live stemc}\label{live-stemc}

\includegraphics{exploration_files/figure-latex/stem carbon-1.pdf}

\includegraphics{exploration_files/figure-latex/unnamed-chunk-3-1.pdf}

\begin{verbatim}
## Warning: Removed 12 rows containing missing values (geom_point).
\end{verbatim}

\includegraphics{exploration_files/figure-latex/starting year on variability-1.pdf}

Note that as the starting year (scenario) increases, we see that the
largest percent increase in stemc see in the years just after harvest,
along with an upwards shift in the minimum average temperatures. This
may explain the initial large standard deviations within the some of the
above figures.

This trend is less pronounced when considering models that have less
variability in their effect on stemc over time. i.e:

\begin{verbatim}
## Warning: Removed 12 rows containing missing values (geom_point).
\end{verbatim}

\includegraphics{exploration_files/figure-latex/starting year on variability rcp45-cnrm-1.pdf}

\includegraphics{exploration_files/figure-latex/temperate comparison-1.pdf}

total plant carbon (Kg/m2) for each climate scenario based on thinning.
precip is cumulative across all 100 years

\begin{table}[H]
\centering
\begin{tabular}{l|r|r|r|r|r|r|r}
\hline
climproj & 0 & 20 & 40 & 60 & 80 & 100 & total\_precip\\
\hline
rcp85-CAN & 18 & 21 & 24 & 26 & 29 & 32 & 249363.0\\
\hline
rcp85-Had & 20 & 22 & 24 & 26 & 28 & 29 & 221864.0\\
\hline
rcp85-CNRM & 17 & 19 & 21 & 22 & 24 & 26 & 252815.4\\
\hline
rcp45-CAN & 14 & 16 & 18 & 20 & 22 & 24 & 235103.8\\
\hline
rcp45-Had & 16 & 17 & 19 & 21 & 22 & 24 & 212379.8\\
\hline
rcp85-MIROC & 14 & 16 & 17 & 19 & 21 & 23 & 202939.3\\
\hline
rcp45-CNRM & 14 & 15 & 17 & 18 & 20 & 21 & 248247.5\\
\hline
rcp45-MIROC & 11 & 13 & 14 & 16 & 18 & 20 & 200268.4\\
\hline
historic & 9 & 10 & 12 & 14 & 15 & 17 & 145613.6\\
\hline
\end{tabular}
\end{table}

\subparagraph{number of years until stemc returns to 90\% of original
levels for each model and thinning
treatment}\label{number-of-years-until-stemc-returns-to-90-of-original-levels-for-each-model-and-thinning-treatment}

\begin{verbatim}
## Warning in min(flux$wyg[flux$climproj == climscen[i] & flux$thin == thin[j]
## & : no non-missing arguments to min; returning Inf

## Warning in min(flux$wyg[flux$climproj == climscen[i] & flux$thin == thin[j]
## & : no non-missing arguments to min; returning Inf

## Warning in min(flux$wyg[flux$climproj == climscen[i] & flux$thin == thin[j]
## & : no non-missing arguments to min; returning Inf

## Warning in min(flux$wyg[flux$climproj == climscen[i] & flux$thin == thin[j]
## & : no non-missing arguments to min; returning Inf

## Warning in min(flux$wyg[flux$climproj == climscen[i] & flux$thin == thin[j]
## & : no non-missing arguments to min; returning Inf

## Warning in min(flux$wyg[flux$climproj == climscen[i] & flux$thin == thin[j]
## & : no non-missing arguments to min; returning Inf

## Warning in min(flux$wyg[flux$climproj == climscen[i] & flux$thin == thin[j]
## & : no non-missing arguments to min; returning Inf

## Warning in min(flux$wyg[flux$climproj == climscen[i] & flux$thin == thin[j]
## & : no non-missing arguments to min; returning Inf

## Warning in min(flux$wyg[flux$climproj == climscen[i] & flux$thin == thin[j]
## & : no non-missing arguments to min; returning Inf

## Warning in min(flux$wyg[flux$climproj == climscen[i] & flux$thin == thin[j]
## & : no non-missing arguments to min; returning Inf

## Warning in min(flux$wyg[flux$climproj == climscen[i] & flux$thin == thin[j]
## & : no non-missing arguments to min; returning Inf

## Warning in min(flux$wyg[flux$climproj == climscen[i] & flux$thin == thin[j]
## & : no non-missing arguments to min; returning Inf
\end{verbatim}

\begin{table}[H]
\centering
\begin{tabular}{l|l|l|l|l|l|l}
\hline
climproj & 0 & 20 & 40 & 60 & 80 & 100\\
\hline
rcp45-Had & 82 & 57 & 41 & 26 & 12 & 0\\
\hline
rcp45-MIROC & >99 & >99 & >99 & 45 & 14 & 0\\
\hline
rcp45-CNRM & >99 & 97 & 58 & 37 & 18 & 0\\
\hline
rcp45-CAN & >99 & 85 & 48 & 28 & 11 & 0\\
\hline
rcp85-Had & 53 & 42 & 32 & 21 & 8 & 0\\
\hline
rcp85-MIROC & >99 & >99 & 49 & 29 & 11 & 0\\
\hline
rcp85-CNRM & 71 & 54 & 40 & 27 & 12 & 0\\
\hline
rcp85-CAN & 65 & 50 & 35 & 22 & 7 & 0\\
\hline
historic & >99 & >99 & >99 & >99 & >99 & 0\\
\hline
\end{tabular}
\end{table}

\begin{center}\rule{0.5\linewidth}{\linethickness}\end{center}

\subsection{streamflow}\label{streamflow}

\includegraphics{exploration_files/figure-latex/unnamed-chunk-5-1.pdf}

\includegraphics{exploration_files/figure-latex/unnamed-chunk-6-1.pdf}

Things to explore here: overlay precipitation w/ ET. can also look out
groundwater to try to figure out where the water is going


\end{document}
